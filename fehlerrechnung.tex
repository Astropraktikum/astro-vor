\documentclass[titlepage]{scrartcl}
% Code Darstellung
\usepackage{listings}
\usepackage{listingsutf8}
\usepackage{multicol}

%lange Tabellen
\usepackage{longtable}
%Referenzen zwischen unterschiedlichen Dateien
\usepackage{xr}
\externaldocument{theorie}
\usepackage{lscape}
%Deutsche Sprachunterstützung
\usepackage[utf8]{inputenc}
\usepackage[ngerman]{babel}
\usepackage{marvosym}
\DeclareUnicodeCharacter{20AC}{\EUR}

%Für das Einbinden von Bildern
\usepackage{graphicx}

%Tabellen
\usepackage{array}

%Tabellen automatisch schoener
\usepackage{booktabs}

%Caption
\usepackage{caption}
\usepackage{subcaption}

%Formeln
\usepackage{mathtools}
\usepackage{amsmath}
\usepackage{amssymb}
\usepackage{amstext}
\usepackage{dsfont}

%\usepackage{mnsymbol}

% Interssante natbib Optionen: 
% numbers : Nummerierte Zitateinheiten
% sort&compress : Bei mehrfachen Zitaten, Sortierung und ggf. Verkürzungen
%\usepackage[]{natbib}

%Vectorpfeile schöner
\usepackage{esvect}

%Formatierung
\usepackage[T1]{fontenc}
\usepackage{lmodern}
\usepackage{microtype}

%Schaltbilder malen
%\usepackage[europeanresistors,cuteinductors,siunitx]{circuitikz}
\usepackage{comment}
\usepackage{csquotes}

%Formatierungsanweisungen
\newcommand{\wichtig}[1]{\underline{\large{#1}}}
\newcommand{\aref}[1]{Abb. \ref{#1}}
\newcommand{\R}{\mathbb{R}}
\newcommand{\K}{\mathbb{K}}
\newcommand{\C}{\mathbb{C}}

%Klickbare Referenzen
\usepackage[hidelinks]{hyperref}

\begin{document}
\title{Vorübungen Fehlerrechnung}
\section{Vorübung 1}
i) $ d = (0.77 \pm 0.09)\ kpc $ \\
ii) $ t = (1.6 \pm 1.3) \ s $ \\
iii) $ \pi = (32.6 \pm 2.5) \cdot 10^{-3} \ arcsec $ \\
iv) $ \lambda = (4861 \pm 5) \AA $ \\
v) $ p = (3.250 \pm 0.05)\cdot  10^{-3} \cdot  g\ cm\ s^{-1} $ \\
vi) $ E = (11.13 \pm 0.29) \  keV $ \\
vii) $ Q = (121.5 \pm 1.1) \cdot 10^{16} \ C $ \\
viii) $ B = (2.4 \pm 0.8)  \cdot 10^{11} T $ \\
ix) $ F = (715 \pm 22)\ Jy $ \\
x) $ \sigma = (200 \pm 100)\ mb $ 

\section{Vorübung 2}
a) \\
Nach den in der Aneitungen angegebenen Formeln ergibt sich: \\
i) $ \bar{log\ g} = \frac{(1.03 + 1.14 + 1.06 + 1.08 + 1.13 + 1.17) \ dex }{6} = 1.102\ (\pm 0.022) \ dex $  \\
ii) $\sigma_{log\ g} = \sqrt{\frac{\Sigma(x_i - \bar{x})^2}{n-1}} = 0.053\ (\pm 0.017) \ dex $ \\
iii) $\delta \bar{log\ g} = \frac{\sigma_{log\ g}}{\sqrt{n}}  = 0.022 \ dex $ \\
iv)  $ \delta \sigma_{log\ g} = \frac{\sigma_{log\ g}}{\sqrt{2n-2}} = 0.017 \ dex $ \\ \\
b) \\
Die nicht-logarithmische Größe ergibt sich offensichtlich durch Umkehrung des Logarithmus: 
\begin{equation}
g = 10^{log\ g\cdot dex^{-1}} \cdot cm s ^{-2}.
\nonumber
\end{equation} 
Somit ergibt sich: \\
i') 
\begin{equation}
 \bar{g} = 10^{\bar{log\ g} \ dex^{-1}} \cdot cm\ s^{-2} = (12.7 \pm 0.7)\ cm\ s^{-2}. 
 \nonumber
\end{equation}
ii') Die in das CGS-System umgerechneten Werte in $cm\ s^{-2} $ ergeben sich unter Angabe hinreichend vieler Nachkommastellen zu: \\
10.71519305, \\
13.80384265, \\
11.48153621, \\
12.02264435, \\
13.48962883, \\
14.79108388. \\ \\
Mittels dieser Werte ergibt sich nach der in a) genannten Formel eine Standardabweichung von 
\begin{equation}
\sigma_{g} = (1.6 \pm 0.5)\ cm\ s^{-2}.
\nonumber
\end{equation}
iii') 
\begin{equation}
\delta \bar{g} = |\frac{d}{d\ log\ g}\ g| \cdot \delta \bar{log\ g} = |[ln{10}\ dex^{-1} \cdot 10^{log\ g\cdot dex^{-1}}]_{\bar{log\ g}}| \cdot \delta \bar{log\ g}\cdot cm s ^{-2} =
\nonumber 
\end{equation}
\begin{equation}
= |ln{10} \cdot 10^{1.102}| \cdot 0.022 \ dex \cdot \frac{cm\ s^{-2}}{dex}  = 0.7\  cm\ s^{-2}.
\nonumber
\end{equation}
iv') Aus der in ii') errechneten Standardabweichung ergibt sich ein Fehler von 
\begin{equation}
\delta \sigma_{g} = \frac{\sigma_{g}}{\sqrt{10}} = 0.5 \ cm\ s^{-2}. 
\nonumber
\end{equation}

Die Fallbeschleunigung ist mit $ 0.13\ \frac{m}{s^2} $ um etwa zwei Größenordnungen kleiner als die Fallbeschleunigung auf der Erde (knapp $ 10\ \frac{m}{s^2} $)! \\ \\
c) Es gilt:  
\begin{equation}
g = 10^{log\ g\cdot dex^{-1}} \cdot cm\ s ^{-2}.
\nonumber
\end{equation} 
Somit folgt nach der Formel für die Fehlerfortpflanzung: 
\begin{equation}
\delta g = |\frac{d}{d\ log\ g}\ g| \cdot \delta log\ g = [|\ln 10 \cdot 10^{log\ g\cdot dex^{-1}} \cdot cm\ s ^{-2} dex^{-1}|]_{log\ g = log\ \bar{g}} \cdot \delta (log\ g)= 
\nonumber
\end{equation}
\begin{equation}
= \ln 10 \cdot \bar{g} \cdot \delta (log\ g) \cdot dex^{-1}. 
\nonumber
\end{equation}
Es gilt deshalb: 
\begin{equation}
\frac{\delta g}{\bar{g}} = \ln 10 \cdot 0.2 = 0.47.  
\end{equation}
Ein solcher Fehler von knapp 50\% des gemessenen Wertes ist verhältnismäßig groß. Es sind also keine genauen Aussagen über die exakte Fallbeschleunigung möglich, die Güte der Messung ist eher gering. 
\end{document}