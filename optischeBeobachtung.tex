\documentclass[titlepage]{scrartcl}
% Code Darstellung
\usepackage{listings}
\usepackage{listingsutf8}
\usepackage{multicol}

%lange Tabellen
\usepackage{longtable}
%Referenzen zwischen unterschiedlichen Dateien
\usepackage{xr}
%\externaldocument{theorie}
\usepackage{lscape}
%Deutsche Sprachunterstützung
\usepackage[utf8]{inputenc}
\usepackage[ngerman]{babel}
\usepackage{marvosym}
\DeclareUnicodeCharacter{20AC}{\EUR}

%Für das Einbinden von Bildern
\usepackage{graphicx}

%Tabellen
\usepackage{array}

%Tabellen automatisch schoener
\usepackage{booktabs}

%Caption
\usepackage{caption}
\usepackage{subcaption}

%Formeln
\usepackage{mathtools}
\usepackage{amsmath}
\usepackage{amssymb}
\usepackage{amstext}
\usepackage{dsfont}

%\usepackage{mnsymbol}

% Interssante natbib Optionen: 
% numbers : Nummerierte Zitateinheiten
% sort&compress : Bei mehrfachen Zitaten, Sortierung und ggf. Verkürzungen
%\usepackage[]{natbib}

%Vectorpfeile schöner
\usepackage{esvect}

%Formatierung
\usepackage[T1]{fontenc}
\usepackage{lmodern}
\usepackage{microtype}

%Schaltbilder malen
%\usepackage[europeanresistors,cuteinductors,siunitx]{circuitikz}
\usepackage{comment}
\usepackage{csquotes}

%Formatierungsanweisungen
\newcommand{\wichtig}[1]{\underline{\large{#1}}}
\newcommand{\aref}[1]{Abb. \ref{#1}}
\newcommand{\R}{\mathbb{R}}
\newcommand{\K}{\mathbb{K}}
\newcommand{\C}{\mathbb{C}}

%Klickbare Referenzen
\usepackage[hidelinks]{hyperref}
\begin{document}
\section{Vorübung 1}
Scheinbare Helligkeitsformel gewichtet mit der Öffnung (der Lichtsammelfläche)
\begin{align*}
m_2-m_1&=-2.5 \cdot\log \frac{\frac{F_2}{D_2^2}}{\frac{F_1}{D_1^2}}\\
\rightarrow m_2&=m1+2.5\cdot\log\frac{D_2^2}{D_1^2}\\
m_2&=m_1+5\cdot \log \frac{D_2}{D_1}\\
m_2&=m_1-5\cdot\log D_1 +5\cdot\log D_2
\end{align*}
Einsetzen von $m_{Grenze,Auge}=6.0$ Magnituden statt $m_1$, Durchmesser Auge $D_{Auge} = 0.8cm$ statt $D_1$, $m_{Grenze}$ statt $m_2$ und Durchmesser $D$ statt $D_2$:
\begin{align*}
m_{Grenze}&=m_{Grenze, Auge}-5\cdot\log D_{Auge} +5\cdot\log D\\
m_{Grenze}&=6.0-(-0.5) +5\cdot\log D\\
m_{Grenze}&=6.5 + 5\cdot \log D
\end{align*}
\subsection{Herleitung}
\subsection{Anwendung}
Die Grenzmagnitude für das kleine Bamberger Teleskop mit Durchmesser D = 40cm
\begin{align*}
m_{Grenze} (40cm)&= 6.5 + 5 \log \frac{40cm}{cm}\\
&= 14.5
\end{align*}

Die Grenzmagnitude für das kleine Bamberger Teleskop mit Durchmesser D = 50cm
\begin{align*}
m_{Grenze} (50cm)&= 6.5 + 5 \log \frac{50cm}{cm}\\
&= 15.0
\end{align*}
\section{Vorübung 2}
\subsection{Jupiter}
Es gilt
\begin{equation*}
B = f_{Teleskop}\cdot \varphi
\end{equation*}
Daraus Folgt für den Jupiter mit einem Winkeldurchmesser von $40"$ und der Brennweite $f_{Teleskop}=3.35m$ des 50cm Teleskops eine Bildgröße von
\begin{align*}
B_{Jupiter}&=3.36m \cdot 40"\\
&= 3,7 cm
\end{align*}
\subsection{Seeing}
Das mittlere Seeing in Bamberg beträgt $\sim3 "$ 
Damit ist die Ausdehnung auf der Brennebene
\begin{align*}
B_{Seeing}&=3.36m \cdot 3"\\
&= 2.79 \cdot 10^3 \mu m
\end{align*}
\end{document}