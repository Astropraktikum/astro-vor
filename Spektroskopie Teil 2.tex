\documentclass[titlepage]{scrartcl}
% Code Darstellung
\usepackage{listings}
\usepackage{listingsutf8}
\usepackage{multicol}

%lange Tabellen
\usepackage{longtable}
%Referenzen zwischen unterschiedlichen Dateien
\usepackage{xr}
%\externaldocument{theorie}
\usepackage{lscape}
%Deutsche Sprachunterstützung
\usepackage[utf8]{inputenc}
\usepackage[ngerman]{babel}
\usepackage{marvosym}
\DeclareUnicodeCharacter{20AC}{\EUR}

%Für das Einbinden von Bildern
\usepackage{graphicx}

%Tabellen
\usepackage{array}

%Tabellen automatisch schoener
\usepackage{booktabs}

%Caption
\usepackage{caption}
\usepackage{subcaption}

%Formeln
\usepackage{mathtools}
\usepackage{amsmath}
\usepackage{amssymb}
\usepackage{amstext}
\usepackage{dsfont}

%\usepackage{mnsymbol}

% Interssante natbib Optionen: 
% numbers : Nummerierte Zitateinheiten
% sort&compress : Bei mehrfachen Zitaten, Sortierung und ggf. Verkürzungen
%\usepackage[]{natbib}

%Vectorpfeile schöner
\usepackage{esvect}

%Formatierung
\usepackage[T1]{fontenc}
\usepackage{lmodern}
\usepackage{microtype}

%Schaltbilder malen
%\usepackage[europeanresistors,cuteinductors,siunitx]{circuitikz}
\usepackage{comment}
\usepackage{csquotes}

%Formatierungsanweisungen
\newcommand{\wichtig}[1]{\underline{\large{#1}}}
\newcommand{\aref}[1]{Abb. \ref{#1}}
\newcommand{\R}{\mathbb{R}}
\newcommand{\K}{\mathbb{K}}
\newcommand{\C}{\mathbb{C}}

%Klickbare Referenzen
\usepackage[hidelinks]{hyperref}
\begin{document}
\section{Vorübung 4}
Es gilt:
\begin{equation}
R_{Spalt} = \frac{n\cdot f_{Koll}}{d\cdot b\cdot cos \alpha}\cdot \lambda
\end{equation}
Ersetze $\lambda$ durch $\lambda_n^0$:
\begin{equation}
R_{Echelle} = \frac{n\cdot f_{Koll}}{d\cdot b\cdot cos \alpha}\cdot \frac{d}{n}[sin \alpha + sin(2\Theta-\alpha)] = \frac{f_{Koll}}{b}\cdot [tan \alpha + \frac{sin(2\Theta-\alpha)}{cos \alpha}]
\label{blabla}
\end{equation}
Setze Tabellenwerte ein:
\begin{equation}
R_{Echelle} = \frac{100 \cdot 10^{-3}m}{25 \cdot 10^{-6}m}\cdot [tan (73.2^{\circ}) + \frac{sin(53.8^{\circ})}{cos (73.2^{\circ})}] \approx 37000
\end{equation}
\section{Vorübung 5}
Ersetze Größen aus Gleichung (\ref{blabla}) durch die kameraseitigen Größen ($b \rightarrow 2b_{Pixel}$) aus dem Nyquist-Kriterium)
\begin{equation}
R_{CCD} = \frac{f_{Kamera}}{2b_{Pixel}}\cdot [tan \beta + \frac{sin(2\Theta-\beta)}{cos \beta}]
\end{equation}
Setze Tabellenwerte ein:
\begin{equation}
R_{CCD} = \frac{150\cdot10^{-3}m}{18\cdot10^{-6}m}\cdot [tan (53.8^{\circ}) + \frac{sin(73.2^{\circ})}{cos (53.8^{\circ})}] \approx 25000
\end{equation}
Die Auflösung der CCD-Kamera ist in der gleichen Größenordnung wie die des Spektrographen.
\end{document}