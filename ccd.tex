\documentclass[titlepage]{scrartcl}
% Code Darstellung
\usepackage{listings}
\usepackage{listingsutf8}
\usepackage{multicol}

%lange Tabellen
\usepackage{longtable}
%Referenzen zwischen unterschiedlichen Dateien
\usepackage{xr}
\externaldocument{theorie}
\usepackage{lscape}
%Deutsche Sprachunterstützung
\usepackage[utf8]{inputenc}
\usepackage[ngerman]{babel}
\usepackage{marvosym}
\DeclareUnicodeCharacter{20AC}{\EUR}

%Für das Einbinden von Bildern
\usepackage{graphicx}

%Tabellen
\usepackage{array}

%Tabellen automatisch schoener
\usepackage{booktabs}

%Caption
\usepackage{caption}
\usepackage{subcaption}

%Formeln
\usepackage{mathtools}
\usepackage{amsmath}
\usepackage{amssymb}
\usepackage{amstext}
\usepackage{dsfont}

%\usepackage{mnsymbol}

% Interssante natbib Optionen: 
% numbers : Nummerierte Zitateinheiten
% sort&compress : Bei mehrfachen Zitaten, Sortierung und ggf. Verkürzungen
%\usepackage[]{natbib}

%Vectorpfeile schöner
\usepackage{esvect}

%Formatierung
\usepackage[T1]{fontenc}
\usepackage{lmodern}
\usepackage{microtype}

%Schaltbilder malen
%\usepackage[europeanresistors,cuteinductors,siunitx]{circuitikz}
\usepackage{comment}
\usepackage{csquotes}

%Formatierungsanweisungen
\newcommand{\wichtig}[1]{\underline{\large{#1}}}
\newcommand{\aref}[1]{Abb. \ref{#1}}
\newcommand{\R}{\mathbb{R}}
\newcommand{\K}{\mathbb{K}}
\newcommand{\C}{\mathbb{C}}

%Klickbare Referenzen
\usepackage[hidelinks]{hyperref}

\begin{document}
\section{Vorübung 1}
Für die Energie eines Photons gilt: 
\begin{equation}
E = h \nu = h \frac{c}{\lambda}. 
\end{equation}
Dementsprechend folgt: 
\begin{equation}
E = 6.626 \cdot 10^{34}\ Js \frac{3.0 \cdot 10^8 \frac{m}{s}}{450\ nm} = 2.8\ eV. 
\end{equation}
Die Energie eines einfallenden Photons reicht also für zwei Elektron/Lochpaare aus. 	Da aufgrund der Quantisierung des Photons sämtliche Energie auf ein gebundenes Elektron übergeht, kann mit einem Phtoton auch nur ein Paar erzeugt werden. 
Für rotes Licht (etwa Wellenlänge $\lambda = 700\ nm $) ergibt eine analoge Rechnung eine Energie von $1.7\ eV $. Unter der Annahme, dass nur die Bandlücke von $1.12\ eV $ aufgebracht werden muss, ist das CCD rot-empfindlich, bei einer Temperatur von $ 300\ K $ (also etwa Raumtemperatur) kann in der Realität aber kein rotes Licht detektiert werden. 



\end{document}