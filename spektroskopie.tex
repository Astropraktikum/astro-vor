\documentclass[titlepage]{scrartcl}
% Code Darstellung
\usepackage{listings}
\usepackage{listingsutf8}
\usepackage{multicol}

%lange Tabellen
\usepackage{longtable}
%Referenzen zwischen unterschiedlichen Dateien
\usepackage{xr}
\externaldocument{theorie}
\usepackage{lscape}
%Deutsche Sprachunterstützung
\usepackage[utf8]{inputenc}
\usepackage[ngerman]{babel}
\usepackage{marvosym}
\DeclareUnicodeCharacter{20AC}{\EUR}

%Für das Einbinden von Bildern
\usepackage{graphicx}

%Tabellen
\usepackage{array}

%Tabellen automatisch schoener
\usepackage{booktabs}

%Caption
\usepackage{caption}
\usepackage{subcaption}

%Formeln
\usepackage{mathtools}
\usepackage{amsmath}
\usepackage{amssymb}
\usepackage{amstext}
\usepackage{dsfont}

%\usepackage{mnsymbol}

% Interssante natbib Optionen: 
% numbers : Nummerierte Zitateinheiten
% sort&compress : Bei mehrfachen Zitaten, Sortierung und ggf. Verkürzungen
%\usepackage[]{natbib}

%Vectorpfeile schöner
\usepackage{esvect}

%Formatierung
\usepackage[T1]{fontenc}
\usepackage{lmodern}
\usepackage{microtype}

%Schaltbilder malen
%\usepackage[europeanresistors,cuteinductors,siunitx]{circuitikz}
\usepackage{comment}
\usepackage{csquotes}

%Formatierungsanweisungen
\newcommand{\wichtig}[1]{\underline{\large{#1}}}
\newcommand{\aref}[1]{Abb. \ref{#1}}
\newcommand{\R}{\mathbb{R}}
\newcommand{\K}{\mathbb{K}}
\newcommand{\C}{\mathbb{C}}

%Klickbare Referenzen
\usepackage[hidelinks]{hyperref}

\begin{document}
\section{Vorübung 1}
a) Aus der Kleinwinkelnäherung folgt: 
\begin{equation}
B = 2\cdot \sin{\frac{\phi}{2}} \cdot f = 2 \cdot \frac{\phi}{2} \cdot f = f \cdot \phi.
\end{equation}
b) Der vom Spalt \enquote{umspannte} Winkelbereich $\Delta \alpha $ beträgt nach der in a) gezeigten Formel 
\begin{equation}
\Delta \alpha = \frac{b}{f_{koll}}.
\end{equation}
c) (10.1) lautet 
\begin{equation}
d(\sin \alpha  + \sin \beta) = n \cdot \lambda. 
\end{equation}
Als Ableitung nach $\alpha$ ergibt sich:
\begin{equation}
\frac{d\lambda}{d\alpha} = \frac{d}{n} \cdot \cos{\alpha}.
\end{equation}
d) Für hinreichend kleine $\alpha$ gilt diese Näherung. Somit ergibt sich unter dieser Bedingung:
\begin{equation}
\Delta \lambda = \frac{d\lambda}{d\alpha} \cdot \Delta \alpha = \frac{d}{n} \cdot \cos{\alpha} \cdot \frac{b}{f_{koll}}.
\end{equation} 
\section{Vorübung 2}
Entsprechend der Vorübung 2 in Kapitel 7 ergibt sich eine Ausdehnung von $139\ \mu m$ in der Fokalebene. Ein Wert von etwa $140\ \mu m$ wäre also der kleinste mögliche Wert mit voller Lichteinstrahlung und somit der ideale Wert für die Blendenöffnung. 

\end{document}